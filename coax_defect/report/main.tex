% ****** Electric Field Concentration in Coaxial HV Cables ******
%
\documentclass[%
 reprint,
 amsmath,amssymb,
 aps,
]{revtex4-2}

\usepackage{graphicx}
\usepackage{dcolumn}
\usepackage{bm}
\usepackage{hyperref}
\usepackage{xcolor}

\begin{document}

\title{Electric Field Concentration in Coaxial HV Cables due to Dielectric Defects: Simulation and Analysis}

\author{Jacob Larsson}
\affiliation{Chalmers University of Technology, Gothenburg, Sweden}

\date{\today}

\begin{abstract}
Insulation defects such as voids and contaminants are a leading cause of failure in high-voltage cables. Industry data indicate that insulation faults account for the majority of internal failures in cross-linked polyethylene (XLPE) cables, making it important to understand how defects modify the local electric field. We investigate how small defects, specifically air bubbles and water-like inclusions, in the insulation of a coaxial high-voltage cable change the electric field distribution and the risk of breakdown. We model a 2D cross-section of the cable and solve the electrostatic equation $\nabla \cdot (\epsilon \nabla V)=0$ numerically on a Cartesian grid using a flux-conserving stencil with harmonic averaging of the permittivity at material interfaces. The results are compared with the analytical field for an ideal, defect-free coaxial cable, $E_0(r)=V_0/[r\ln(R_\mathrm{out}/R_\mathrm{in})]$. We find that the global peak field (at the inner electrode interface) changes by less than 1\% when adding a small defect at mid-radius. Locally at the defect position, the peak field inside an air void reaches about $1.45$~kV/mm, while a high-permittivity inclusion increases the peak field in the surrounding XLPE to about $1.74$~kV/mm (about 20\% above the defect-free value at the same radius). A larger void placed closer to the inner conductor can exceed a partial-discharge inception field for millimetre-scale gaps, highlighting how important defect position is.

\begin{description}
\item[Project Topic]
B
\item[Teaching Assistant]
Agnese Callegari
\end{description}
\end{abstract}

\maketitle


%=============================================================================
\section{\label{sec:intro}Introduction}
%=============================================================================

Coaxial and power cables use solid polymer insulation, typically cross-linked polyethylene (XLPE) with relative permittivity $\epsilon_r \approx 2.3$, to maintain a controlled electric field between the high-voltage conductor and the grounded shield~\cite{kuffel2000highvoltage}. In an ideal, defect-free cable the electric field is purely radial and follows the analytical formula
\begin{equation}
E_0(r) = \frac{V_0}{r\ln(R_\mathrm{out}/R_\mathrm{in})},
\label{eq:ideal_coax}
\end{equation}
where $V_0$ is the applied voltage and $R_\mathrm{in}$, $R_\mathrm{out}$ are the inner and outer conductor radii~\cite{jackson1999classical}. The field is strongest at the inner conductor surface and decreases with increasing radius.

In practice, however, manufacturing imperfections, ageing, and moisture can create localised regions with permittivity very different from the insulation. Common examples include air-filled voids ($\epsilon_r \approx 1$) and water-filled pockets ($\epsilon_r \approx 80$)~\cite{nordling2020physics}. Because the electric field inside a low-permittivity void is increased relative to the surrounding dielectric, such defects are known to initiate partial discharges (PD), i.e., localised breakdowns that do not immediately bridge the insulation but erode it over time, eventually leading to complete failure~\cite{dissado1992electrical,kuffel2000highvoltage}. Industry surveys confirm that insulation defects are a major reliability concern: according to CIGRE working-group reports, insulation-system faults account for roughly 64\% of internal failures in extruded HV cables~\cite{cigre2009}.

Considerable research has been made to understand the field enhancement caused by voids. For a spherical cavity in a uniform dielectric, electrostatics predicts an enhancement factor $f = 3\epsilon_r/(2\epsilon_r+1)$, which for XLPE ($\epsilon_r = 2.3$) gives $f \approx 1.23$~\cite{crichton1989pdmodel}.

In this project we develop a finite-volume solver on a Cartesian grid to study a 2D cross-section of a coaxial cable containing a single circular defect. We verify the solver against the analytical solution~\eqref{eq:ideal_coax}, then examine two defect scenarios: (i) an air bubble ($\epsilon_r = 1$) and (ii) a water-like inclusion ($\epsilon_r = 80$). We extract peak fields in the XLPE and inside the defect, compare them to typical PD inception values from the literature~\cite{meek1940breakdown,kuffel2000highvoltage}, and discuss the implications for cable reliability.


%=============================================================================
\section{\label{sec:overview}Overview}
%=============================================================================

Several numerical techniques are available for solving electrostatic problems with spatially varying permittivity. Table~\ref{tab:methods} summarises the main options and their suitability for the present study.

\begin{table*}
  \caption{{\bf Overview of modeling options for coaxial insulation with defects.}}
  \label{tab:methods}
  \begin{tabular}{|c|c|c|c|}
    \hline
    {\bf Method} & {\bf Use case} & {\bf Features} & {\bf Suitability} \\
    \hline
    Analytical coax & Homogeneous $\epsilon$ & Closed-form $E_0(r)$; no defects & Baseline only \\
    \hline
    FD (node-based) & Structured grids & Simple stencils; discontinuous $\epsilon$ needs care & OK with special treatment \\
    \hline
    FVM (face fluxes) & Heterogeneous media & Conserves flux; harmonic-mean $\epsilon$ at faces~\cite{eymard2000fvm} & Very good for $\epsilon$-jumps \\
    \hline
    FEM (unstructured) & Complex geometries & Flexible meshing; robust material interfaces & Very good; highest flexibility \\
    \hline
    BEM & Piecewise homogeneous & Reduces dimensionality; interfaces explicit & Good if few regions \\
    \hline
  \end{tabular}
\end{table*}

\textbf{Analytical solution.}
For a homogeneous coaxial geometry the potential and field can be written in closed form (Eq.~\ref{eq:ideal_coax}). This provides an essential baseline for code verification and for quantifying how much defects perturb the field.

\textbf{Finite Difference Method (FD).}
FD approximates derivatives by differences between neighbouring grid values. It is straightforward to implement on uniform Cartesian grids but can produce spurious field jumps at material interfaces unless the discretisation accounts for the permittivity discontinuity~\cite{taflove2005fdtd}.

\textbf{Finite Volume Method (FVM).}
FVM discretises the domain into control volumes and keeps track of how much flux passes through each face. At a face between two materials the flux $\epsilon\,\partial V/\partial n$ must be continuous; using a harmonic average of the two permittivities naturally satisfies this condition~\cite{eymard2000fvm}. FVM is therefore well suited to problems with embedded inclusions.

\textbf{Finite Element Method (FEM).}
FEM divides the domain into triangular or quadrilateral elements and solves a weak (integral) form of the governing equation. It handles complex geometries and material interfaces elegantly, but requires mesh generation software and more involved implementation.

\textbf{Boundary Element Method (BEM).}
BEM reduces the problem to integrals over material interfaces only, which can be efficient when there are few homogeneous regions. For multiple or arbitrarily placed defects, however, the bookkeeping becomes cumbersome.

\textbf{Choice of method.}
We adopt a finite-volume-inspired approach on a Cartesian grid. This combines the simplicity of structured grids with correct handling of $\epsilon$-jumps (via flux continuity) and avoids the need for an external mesh generator.


%=============================================================================
\section{\label{sec:method}Method}
%=============================================================================

\begin{figure*}
    \centering
    \includegraphics[width=0.95\textwidth]{figures/method_overview.png}
    \caption{\textbf{Model geometry and typical output fields.} Example visualisation for the defect-free coaxial cable. Left: equipotential contours. Middle: $|\mathbf{E}|$ magnitude (log scale). Right: field-line streamplot. Circular Dirichlet regions represent the inner conductor at $V_0$ and the grounded outer conductor. Defect cases are created by assigning a different $\epsilon_r$ inside a small circular region in the XLPE insulation.}
    \label{fig:method}
\end{figure*}

Figure~\ref{fig:method} gives an overview of the model geometry, boundary conditions, and the main post-processed field visualisations used throughout the report.

\subsection{Geometry and boundary conditions}

We consider a 2D cross-section of a coaxial cable with inner conductor radius $R_\mathrm{in}=2~mm$ held at $V_0=15~kV$, and outer conductor radius $R_\mathrm{out}=10~mm$ grounded ($V=0$). The annular region between the conductors is filled with XLPE ($\epsilon_r=2.3$). A single circular defect is placed at the midpoint of the insulation thickness, i.e.\ at radial position $r_c = (R_\mathrm{in}+R_\mathrm{out})/2 = 6~mm$. Two defect types are studied:
\begin{enumerate}
    \item \textbf{Air bubble:} $\epsilon_r = 1$, radius $r_d=0.5~mm$ (simulating a void).
    \item \textbf{Water-like inclusion:} $\epsilon_r = 80$, radius $r_d=0.3~mm$ (simulating moisture ingress).
\end{enumerate}
A baseline case with no defect (homogeneous XLPE) is included for validation.

The computational domain is a square of side $L=2\cdot 1.06\,R_\mathrm{out}=21.2~mm$ centred on the cable axis. The outer edges of the square are given zero-flux (Neumann) boundary conditions to approximate open space. Nodes whose distance from the origin falls within a small tolerance of $R_\mathrm{in}$ or $R_\mathrm{out}$ are marked as Dirichlet nodes and assigned the conductor potentials. This ``staircase/band'' representation of circular electrodes on a Cartesian grid introduces a small geometric error localised near the interfaces; its impact is assessed by comparison to the analytical coax solution in Sec.~\ref{sec:results}.

\subsection{Main equation}

In the electrostatic limit the potential $V(x,y)$ satisfies
\begin{equation}
\nabla \cdot \bigl(\epsilon(x,y)\,\nabla V\bigr) = 0,
\label{eq:laplace}
\end{equation}
where $\epsilon(x,y)$ is the permittivity. At interfaces between materials, $V$ is continuous and the normal component of $\epsilon\nabla V$ (the displacement flux) is also continuous.

\subsection{Finite-volume-inspired discretisation}

We place a uniform Cartesian grid over the 2D cross-section and treat the voltage $V$ at the grid points as unknowns. For points that are not on the electrodes there is no free charge, so the net ``electric flux'' leaving a small box around the point should be zero. In practice, this means each interior voltage becomes a weighted average of its four nearest neighbours, where the weights depend on how easily the field can pass through the material.

The most important detail is what happens at a material boundary (e.g.\ XLPE--air): the voltage is still continuous, but the amount of flux crossing the interface must also be consistent. To get this we assign an effective permittivity between two neighbouring cells using harmonic mean
\begin{equation}
\epsilon_\mathrm{face} = \frac{2\,\epsilon_L\,\epsilon_R}{\epsilon_L + \epsilon_R},
\label{eq:harmonic}
\end{equation}
where $\epsilon_L$ and $\epsilon_R$ are the permittivities on the two sides of that face. Using $\epsilon_\mathrm{face}$ helps prevent weird jumps in flux at sharp permittivty changes and is commonly used in heterogeneous-media discretisations~\cite{eymard2000fvm}. Once this rule is applied everywhere, we iteratively update the voltages until they stop changing.

\subsection{Grid resolution}

All simulations use a $720\times720$ grid in the square domain, corresponding to a uniform spacing of $\Delta x \approx 0.0295~mm$. This gives roughly 34 nodes across the 1.0~mm diameter air bubble and about 20 nodes across the 0.6~mm diameter inclusion, which is good for resolving the local perturbations shown in Fig.~\ref{fig:bubble_zoom} and Fig.~\ref{fig:inclusion_zoom}.

\subsection{Solver}

The discretisation creates a large set of coupled equations (one per interior grid point). We solve them iteratively using successive over-relaxation (SOR), i.e.\ a Gauss--Seidel update with relaxation factor $\omega$~\cite{saad2003iterative}. Dirichlet nodes are held fixed at every iteration while all other nodes are updated until the maximum change per iteration falls below a our tolerance value.

\subsection{Post-processing}

Once $V$ is known, the electric field components are computed by central differences:
\[
E_x \approx -\frac{\partial V}{\partial x}, \qquad E_y \approx -\frac{\partial V}{\partial y},
\]
and the magnitude is $|\mathbf{E}| = \sqrt{E_x^2 + E_y^2}$. The values we want  and that we extract are the following:
\begin{itemize}
    \item $|E|_\mathrm{max,XLPE}$: peak field in the XLPE insulation.
    \item $|E|_\mathrm{max,defect}$: peak field inside the defect (for air bubbles).
    \item $|E|_\mathrm{max,ring}$: peak field in the XLPE immediately surrounding the defect (for high-$\epsilon$ inclusions).
    \item Enhancement ratio $\eta = |E|_\mathrm{max}/E_0(R_\mathrm{in})$ relative to the ideal coax.
\end{itemize}

\subsection{Breakdown and PD criteria}

For the XLPE insulation the relevant figure is the peak electric field in the solid insulation. For air-filled voids, partial discharge can initiate when the field inside the void exceeds a inception field. At atmospheric pressure and gap sizes of order 1~mm, a commonly cited threshold is $E_\mathrm{inc}\approx3~kV/mm$~\cite{meek1940breakdown,kuffel2000highvoltage}. If the simulated peak is well below this, PD most likely won't happen but if it's close or higher, PD becomes a possibility.


%=============================================================================
\section{\label{sec:results}Results and Discussion}
%=============================================================================

\subsection{Validation: homogeneous coaxial cable}

Before introducing defects we verify the solver against the analytical solution. Figure~\ref{fig:baseline} shows the radial profile of $|E|(r)$ extracted along a line from the simulation compared to~\eqref{eq:ideal_coax}. The simulated curve follows the expected $1/r$ dependence over the insulation. We get some local artefacts near the electrodes. Away from this region, the agreement is good enough for the comparative defect study presented below and gives us a baseline field at mid-radius of about $|E|\approx 1.3$--$1.4~kV/mm$.

\begin{figure}
    \centering
    \includegraphics[width=\columnwidth]{figures/baseline_radial.png}
    \caption{\textbf{Validation against analytical solution.} Radial profile of $|E|(r)$ for the homogeneous coaxial cable (no defect). Solid line: simulation; dashed line: Eq.~\eqref{eq:ideal_coax}.}
    \label{fig:baseline}
\end{figure}

\subsection{Air bubble ($\epsilon_r=1$)}

Figure~\ref{fig:bubble_zoom} shows the potential distribution and field magnitude for an air bubble of radius $r_d=0.5~mm$ at mid-radius. The global maximum field remains at the inner electrode interface and is essentially unchanged from the baseline (change $<1\%$), which is expected since the small defect is located far from the highest-field region.

\begin{figure}
    \centering
    \includegraphics[width=\columnwidth]{figures/bubble_zoom.png}
    \caption{\textbf{Air bubble zoomed.} Left: potential contours. Middle: $|E|$ magnitude. Right: streamlines.}
    \label{fig:bubble_zoom}
\end{figure}

Locally, however, the picture is different. At the defect position the undisturbed field magnitude is $|E|_0(r_c)\approx 1.3$--$1.4~kV/mm$. Inside the air bubble the peak field reaches $|E|_\mathrm{max,bubble}\approx1.45~kV/mm$ (Table~\ref{tab:summary}). This enhancement is consistent with the classical expectation that low-permittivity voids increase the internal field relative to the surrounding dielectric~\cite{crichton1989pdmodel}.

Comparing to the PD inception threshold $E_\mathrm{inc}\approx3~kV/mm$, we see that $|E|_\mathrm{max,bubble}$ is well below this value. Partial discharge inside this mid-radius bubble is therefore unlikely under the present operating conditions. However, PD risk is strongly position-dependent; a near-electrode case is discussed in Sec.~\ref{sec:sensitivity}.

Figure~\ref{fig:bubble_radial} shows the radial field profile along a line passing through the bubble. The profile matches the analytical curve except for a localised spike at the defect location.

\begin{figure}
    \centering
    \includegraphics[width=\columnwidth]{figures/bubble_radial.png}
    \caption{\textbf{Air bubble radial profile.} $|E|(r)$ along a line through the bubble centre. The dashed line is the analytical profile for the homogeneous cable.}
    \label{fig:bubble_radial}
\end{figure}

\subsection{Water-like inclusion ($\epsilon_r=80$)}

Figure~\ref{fig:inclusion_zoom} shows results for a high-permittivity inclusion ($\epsilon_r=80$) of radius $r_d=0.3~mm$ at mid-radius, such as a water-like inclusion. The inclusion changes the equipotential lines and deflects the field lines locally, leading to a redistribution of stress in the surrounding XLPE.

\begin{figure}
    \centering
    \includegraphics[width=\columnwidth]{figures/inclusion_zoom.png}
    \caption{\textbf{Water-like inclusion zoomed.} Left: potential contours crowd around the inclusion boundary. Middle: $|E|$ magnitude. Right: streamlines.}
    \label{fig:inclusion_zoom}
\end{figure}

The most visible effect is a field concentration in the surrounding XLPE. In a narrow ring around the inclusion the peak field rises to $|E|_\mathrm{max,ring}\approx1.74~kV/mm$ (Table~\ref{tab:summary}), about 20\% above the defect-free peak in the same region. This local increase in the dielectric is relevant for ageing and treeing (cracks), which are caused by stress in the polymer~\cite{dissado1992electrical,kuffel2000highvoltage}.

The global peak near the inner conductor is again unchanged (change $<1\%$ from baseline).

\subsection{\label{sec:sensitivity}Sensitivity to defect position}

To illustrate that defect position can be important regarding PD risk, we consider a more severe scenario: a large air void of radius $r_d=1.5~mm$ placed close to the inner conductor. Figure~\ref{fig:large_bubble_zoom} shows that the equipotential lines and streamlines are strongly distorted and that the field inside the void increases substantially compared to the mid-radius case. The peak field inside the void reaches $|E|_\mathrm{max,void}\approx 4.53~kV/mm$ (Table~\ref{tab:summary}), which exceeds the representative inception field $E_\mathrm{inc}\approx3~kV/mm$. While this scenario is more extreme than the mid-radius bubble, it demonstrates the expected trend: voids closer to the high-field electrode are more likely to trigger PD.

\begin{figure}
    \centering
    \includegraphics[width=\columnwidth]{figures/large_bubble_zoom.png}
    \caption{\textbf{Large void close to the inner conductor.} Zoom near a $r_d=1.5~mm$ air bubble placed near the inner electrode. The field inside the cavity becomes significantly larger than in the mid-radius case which indicates increased PD risk.}
    \label{fig:large_bubble_zoom}
\end{figure}

\begin{table}
  \caption{\textbf{Summary of key field metrics.} ``Global pk'' is the maximum nodal $|\mathbf{E}|$ in the domain. ``Local pk'' is the maximum $|\mathbf{E}|$ in the defect or in the surrounding XLPE ring (for the inclusion).}
  \label{tab:summary}
  \begin{ruledtabular}
  \begin{tabular}{lcccc}
    \textbf{Case} & $\epsilon_r$ & $r_d$ (mm) & Global pk & Local pk \\
                 &             &            & (kV/mm)   & (kV/mm)  \\
    \hline
    Baseline                         & -- & --  & 6.03 & --   \\
    Bubble (mid-radius)              & 1  & 0.5 & 6.03 & 1.45 \\
    Inclusion (mid-radius)           & 80 & 0.3 & 6.04 & 1.74\textsuperscript{a} \\
    Bubble (near inner conductor)    & 1  & 1.5 & 5.42 & 4.53 \\
  \end{tabular}
  \end{ruledtabular}

  \vspace{0.5ex}
  \noindent\textsuperscript{a}``Local pk'' refers to the peak in the XLPE ring around the inclusion.
\end{table}

\subsection{Model limitations and uncertainty}

The present study is intentionally simplified to isolate the electrostatic field redistribution caused by permittivity defects. First, the model is two-dimensional and represents an infinite-length cable with a perfectly uniform cross-section; real defects are three-dimensional and may have irregular shapes and rough interfaces that can further increase/change the field. Second, the analysis is purely electrostatic and does not include space-charge accumulation, temperature dependence etc. These effects can change both the effective field distribution and the long-term ageing behaviour in XLPE insulation~\cite{dissado1992electrical,kuffel2000highvoltage}.

On the numerical side, circular electrodes are represented on a Cartesian grid through Dirichlet ``bands''. This discretisation introduces local artefacts in $|\mathbf{E}|$ very close to the electrode interface (visible in Fig.~\ref{fig:baseline}); for this reason, the physically relevant comparisons in this report focus on (i) radial profiles away from the electrode band and (ii) local peak values in and around the defects. Finally, since defect peaks can be sensitive to resolution when the defect is only a few grid cells across, we chose grid parameters that resolve each defect diameter by many grid points (tens of nodes).


%=============================================================================
\section{\label{sec:conclusion}Conclusions and Outlook}
%=============================================================================

We have developed a finite-volume solver for the 2D electrostatic problem in a coaxial cable cross-section with varying permittivity. The solver was validated against the analytical solution for a homogeneous cable and then used to study two types of insulation defects: air bubbles and water-like inclusions.

The main findings are:
\begin{enumerate}
    \item \textbf{Global field:} For a 15~kV, 2--10~mm coaxial cable, the peak field occurs at the inner electrode interface and is essentially unchanged ($<1\%$) by adding a small defect at mid-radius.
    
    \item \textbf{Air bubbles:} A 0.5~mm air void at mid-radius produces a local peak field of about 1.45~kV/mm, which is well below a representative PD inception field ($\sim$3~kV/mm). In contrast, a larger 1.5~mm void placed closer to the inner conductor reaches about 4.53~kV/mm inside the cavity, indicating that PD risk is dominated by defect position relative to the high-field electrode.
    
    \item \textbf{Water-like inclusions:} A high-permittivity inclusion ($\epsilon_r=80$) increases the peak field in a surrounding XLPE ring to about 1.74~kV/mm (about 20\% above the defect-free value at the same radius). Such local stress concentrations in the polymer can contribute to long-term ageing even when the global field is unaffected.
\end{enumerate}

These results confirm that small, isolated defects at mid-radius have a limited effect on the overall field distribution, but can create localised stress concentrations that may initiate degradation mechanisms over time.


\section{\label{sec:contributions}Contributions}

All modeling, implementation, simulations, post-processing, and writing were carried out by the author.

\section{\label{sec:conflictofinterest}Conflict of interest}

The author declares no conflict of interest.

\section{\label{sec:dataandcodeavailability}Data and code availability}

The code used to generate the results and the figures, along with the simulation outputs, are available in the project repository accompanying this report.






%=============================================================================
\bibliography{biblio}
%=============================================================================

\end{document}
