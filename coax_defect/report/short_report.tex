\documentclass[ reprint,
 amsmath,amssymb,
 aps,
]{revtex4-2}

\usepackage{graphicx}
\usepackage{dcolumn}
\usepackage{bm}
\usepackage{hyperref}
\usepackage{xcolor}

\begin{document}

\title{Electric Field Concentration in Coaxial HV Cables due to Dielectric Defects: Simulation and Analysis}

\author{Jacob Larsson}
\date{\today}

\begin{abstract}
We investigate how localized dielectric defects, like air bubbles and high-permittivity inclusions, in the insulation of a coaxial high-voltage cable affect the electric field and breakdown risk. A 2D cross-section model of the cable is solved numerically and compared to the analytical solution for an ideal, defect-free coaxial cable. The main finding is that air bubbles create strong field concentration around them, which increases breakdown risk. Water droplets also change the field pattern but in a different way.
\begin{description}
\item[Project Topic]
B
\item[Teaching Assistant]
Agnese Callegari
\end{description}
\end{abstract}

\maketitle

\section{\label{sec:intro}Introduction}

\noindent Coaxial and power cables use solid polymer insulation, e.g. XLPE with relative permittivity \(\epsilon_r \approx 2.3\), to keep the electric field under control. In practice, manufacturing imperfections, ageing and moisture can create smaller regions with very different permittivity, such as air bubbles (\(\epsilon_r \approx 1\)) or water-filled pockets (\(\epsilon_r \sim 80\)). These inhomogeneities can disturb the otherwise smooth radial field of an ideal coaxial cable, for which the analytical field is
\(E_0(r) = V_0/[r\ln(R_\mathrm{out}/R_\mathrm{in})]\) \cite{jackson1999classical}, and can locally increase \(|E|\) enough to trigger partial discharges or even breakdown. Understanding how such defects modify the field is important for cable design and condition assessment in high-voltage systems \cite{kuffel2000highvoltage}.

This work presents a numerical study of a 2D cross-section of a coaxial cable with embedded defects. We solve the electrostatic problem with a spatially varying permittivity \(\epsilon(x,y)\) and first verify the solution against the analytical, defect-free coax case. We then compute the electric field, extract peak values in the XLPE and inside the defects, and compare these to typical breakdown and partial-discharge fields reported in the literature \cite{meek1940breakdown,kuffel2000highvoltage}.

\section{\label{sec:overview}Overview}

\noindent We consider four modelling options relevant to electrostatics in heterogeneous dielectrics: analytical coax (homogeneous), finite difference (FD), finite volume (FVM), finite element (FEM), and boundary element (BEM). Table~\ref{tab:methods} summarises typical use cases.

\begin{table*}
  \caption{{\bf Overview of modelling options for coaxial insulation with defects.}}
  \label{tab:methods}
  \begin{tabular}{|c|c|c|c|}
    \hline
    {\bf Method} & {\bf Use case} & {\bf Features} & {\bf Suitability} \\
    \hline
    Analytical coax & Homogeneous \(\epsilon\) & Closed-form \(E_0(r)\); no defects & Baseline only \\
    \hline
    FD (node-based) & Structured grids & Simple stencils; discontinuous \(\epsilon\) needs care & OK with special treatment \\
    \hline
    FVM (face fluxes) & Heterogeneous media & Conserves flux; special averaging of \(\epsilon\) at faces \cite{eymard2000fvm} & Excellent for material interfaces \\
    \hline
    FEM (unstructured) & Complex geometries & Flexible meshing; robust material interfaces & Excellent; highest flexibility \\
    \hline
    BEM & Piecewise homogeneous & Reduces dimensionality; interfaces explicit & Good if few regions \\
    \hline
  \end{tabular}
\end{table*}

\noindent \textbf{Analytical solution.} For a homogeneous coaxial cable without defects, the electric field can be calculated exactly as \(E_0(r) = V_0/[r\ln(R_\mathrm{out}/R_\mathrm{in})]\). This gives us a baseline to validate our numerical method and to measure how much the defects change the field.

\textbf{Finite Difference Method (FD).} This is the simplest numerical approach where we approximate derivatives using differences between neighbouring grid points. It works well for uniform materials but FD can give incorrect field jumps at material interfaces unless it is modified carefully.

\textbf{Finite Volume Method (FVM).} Instead of working with point values, FVM divides the domain into small volumes and ensures that electric flux is conserved across each volume. This naturally handles material interfaces by using suitable averaging of material properties at the boundaries between volumes. This is particularly good for our problem with embedded defects \cite{eymard2000fvm}.

\textbf{Finite Element Method (FEM).} FEM is very flexible and can handle complex geometries by dividing the domain into triangular or quadrilateral elements. It incorporates material interfaces and boundary conditions, but requires more advanced mesh generation.

\textbf{Boundary Element Method (BEM).} BEM only discretises the boundaries between different materials rather than the entire domain. This can be efficient when there are just a few homogeneous regions. For a simple air bubble in XLPE, BEM could work well, but it becomes more complex as we add more defects or vary their positions.

For this study, we choose a finite-volume approach on a Cartesian grid since it provides a good balance. It handles material interfaces correctly through flux conservation, does not require complex mesh generation, and is straightforward to implement for our circular geometry with embedded defects.

\section{\label{sec:method}Method}

\noindent We model the coaxial cross-section with inner conductor at \(V_0\) and outer conductor at 0~V. The dielectric part has \(\epsilon_r = 2.3\). A circular defect is placed at mid-radius with two cases: (i) air bubble, \(\epsilon_r = 1\); (ii) inclusion, \(\epsilon_r = 80\) \cite{nordling2020physics}. The outer edges of the rectangular domain are placed far from the cable and are given zero normal field (no-flux) boundary conditions to mimic open space.

We solve for the potential \(V(x,y)\) in the cable cross-section from
\[
\nabla \cdot \big(\epsilon(x,y)\,\nabla V\big) = 0,
\]
where \(\epsilon(x,y)\) is the relative permittivity (XLPE in the annulus, air or water-like material inside the defect, and \(\epsilon = 1\) outside). Where materials meet, the potential is continuous and the normal component of \(\epsilon\nabla V\) is also continuous.

We work on a non-uniform Cartesian grid. At each interior grid point we relate the potential to the values at its neighbouring points using finite differences. This gives one linear equation for each unknown \(V_{i,j}\), and together these equations form the system that we solve numerically. Across material boundaries we use a simple averaging of the permittivities on each side of the interface \cite{eymard2000fvm}, which helps keep the field behaviour reasonable at the boundary. Grid points on the conductors are set to their fixed voltages \(V_0\) or 0.

Because the computational domain is rectangular, we mark all grid nodes whose distance from the cable centre is close to \(R_\mathrm{in}\) or \(R_\mathrm{out}\). These nodes are assigned \(V_0\) and 0, respectively, instead of interior equations. In this way the electrodes keep their circular shape even though the underlying grid is rectangular.

To resolve sharp field gradients near the electrodes and the defect without making the grid unnecessarily large, we use a mildly stretched grid. The stretching clusters nodes around \(R_\mathrm{in}\), \(R_\mathrm{out}\) and the cable centre. In all runs we ensure there are at least 20--30 grid points across the defect diameter and across the insulation thickness, and we checked that refining the grid further changes the peak field by less than a few percent.

Putting all equations together gives a large linear system \(A v = b\) for the unknown nodal potentials. Most entries in \(A\) are zero because each grid point only couples to its neighbours. We solve this system with a standard conjugate-gradient iterative solver. Once \(V\) is known, the electric field components are obtained from finite differences, \(E_x \approx -\partial V/\partial x\) and \(E_y \approx -\partial V/\partial y\), and the magnitude is \(|E| = \sqrt{E_x^2 + E_y^2}\).

\bibliography{biblio}

\end{document}
